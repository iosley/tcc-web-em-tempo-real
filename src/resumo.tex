{
	\small
	\ignorespacesafterend
	\noindent

	\textbf{Resumo}:
	% Introdução
	Em poucos anos após a sua criação, a World Wide Web passou de páginas estática e não interativa para aplicações ricas e complexas. Estas mudanças foram acompanhadas da melhoria dos ambientes de execução e tecnologias de desenvolvimento. A medida que as aplicações web se tornaram mais dinâmicas, com um alto volume de mensagens trocadas entre clientes e servidores, foi se fazendo necessário o uso de técnicas para reduzir a latência e permitir uma comunicação mais atempada.
	% Objetivo:
	% O objetivo é em geral a informação de maior interesse: qual o objetivo dos autores com esse estudo? Por esse motivo, a primeira frase do artigo deve conter, na minha opinião, o objetivo do estudo, também chamado statement of purpose: “O objetivo deste estudo é…”, “Neste artigo propomos…”, etc.
	Este estudo tem por objetivo compreender as limitações do protocolo HTTP para comunicação em tempo real na web, exibir métodos paliativos no contexto histórico e soluções atualmente utilizadas para desenvolvimento de aplicações web que exijam uma comunicação constante entre cliente e servidor,
	% Materiais e métodos:
	% Com esse objetivo em mente, como os autores propõem atacar o problema? O resumo deve, portanto, dar uma breve descrição da metodologia usada para atingir o objetivo proposto.
	através de uma pesquisa documental com a utilização de materiais teóricos que abordam o tema e desenvolvem conceitos, ideias e entendimentos, visando buscar soluções, e compreender os pontos positivos e negativos de cada solução.
	% Resultados:
	% Deve-se naturalmente apresentar os principais resultados de forma resumida e concreta, através de informações quantitativas úteis em lugar de afirmações vagas de valor dúbio.
	Demonstrar as capacidades e limitações do AJAX, as vantagens da utilização das APIs de Server-Sent Event e WebSockets nos navegadores modernos,
	% Conclusões:
	% Por fim, quais são as conclusões do estudo? Qual a relevância dos resultados apresentados? Como os resultados avançam o conhecimento na área ou ajudam a resolver o problema proposto?
	confirmando que, por ter baixa latência e boa aceitação por parte dos navegadores modernos, o WebSockets se mostra como a melhor solução para a comunicação bi-direcional entre um cliente e servidor, permitindo a comunicação por um único soquete TCP operando na mesma porta padrão do HTTP
	
	\vspace{\onelineskip}

	\textbf{Palavras-chave}: Ajax. XHR. SSE. WebSockets.
}