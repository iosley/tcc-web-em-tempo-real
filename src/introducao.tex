\section{Introdução}

% Focaliza-se a “questão” na literatura para indicar a originalidade e relevância do trabalho, a metodologia utilizada na pesquisa, bem como identificar os objetivos do estudo no ultimo parágrafo.

A internet\footnote{Conjunto mundial de redes de computadores.} como é hoje passou por diversas transformações com a melhoria dos hardwares\footnote{Parte física de um computador} necessários para seu funcionamento e a definição de protocolos formais para a intercomunicação entre diversos dispositivos \cite{Aghaei2012}.

Porem, a transformação mais notável desde então, foi no tratamento dos dados. Hoje as informações estão acessíveis de qualquer lugar do mundo, isto trouxe também um aumento na retenção da informação e uma mudança drástica na maneira de manipular e transmitir estas informações \cite{Leiner2009}.

Hoje as páginas web são mais dinâmicas podendo ser desenvolvidas aplicações modernas, que antes precisavam ser instaladas no computador, rodando direto do navegador\footnote{Programa de computador que possibilita a interação com documentos virtuais da Internet.},  proporcionando a mesma experiência experimentada com as aplicações nativas \cite{Garrett2005}.

Para ter uma experiência fluida nestas aplicações, os navegadores precisam se comunicar com os servidores\footnote{Sistema de computação centralizada que fornece serviços a uma rede de computadores.} de maneira rápida e eficiente. Para isto, são usadas técnicas para a transmissão e recebimento de informações em tempo real. Estas técnicas mantem as informações sincronizadas entra a aplicação cliente e o servidor de maneira transparente para o usuário \cite{offutt2002quality}.

Estes avanços proporcionaram uma participação muito maior dos usuários que agora deixam de ser meros consumidores de conteúdo e passam a ser os maiores produtores de conteúdo multimídia \cite{Aghaei2012}.

No entanto, entende-se que mudanças foram necessárias para acompanhar esta evolução. Mudanças que pudessem coexistir com a maneira já existente de transmissão de dados ao passo que tornasse mais eficiente e mais rápida a transmissão da informação.

