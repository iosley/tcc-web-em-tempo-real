\section{METODOLOGIA}

%o que eu li
O estudo aborda conhecimento a respeito da comunicação em tempo real entre servidores e clientes de aplicações web, para isso se fez necessário direcionar a abordagem em base da utilização de material teórico por meio de uma pesquisa documental, que segundo \cite{Marconi2003}, tem documentos como fonte primária de informação.

%por que escolhi
Buscando uma melhor análise histórica do desenvolvimento de aplicações em tempo real para a web, este trabalho é pautado na investigação qualitativa a respeito do tema proposto, desenvolvendo conceitos, ideias e entendimentos a partir de fontes secundárias \cite{prodanov2013metodologia}.

Do ponto de vista de sua natureza, está é uma pesquisa básica pois tem como objetivo gerar conhecimentos úteis para o avanço da ciência sem aplicação prática prevista \cite{prodanov2013metodologia}.

%onde eu li
Visando compreender as limitações e em busca por soluções para comunicação em tempo real na web, este estudo tem cunho exploratório pois proporciona maior familiaridade com o tema, envolvendo levantamento bibliográfico e documental \cite{gil2002elaborar}.