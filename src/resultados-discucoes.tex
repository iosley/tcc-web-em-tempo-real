\section{Resultados e Discuções}


%pré javascript

Nos primeiros dias da World Wide Web, um navegador precisava apresentar apenas alguns tipos de dados aos usuários. Para não ser limitado por esses tipos de dados, os desenvolvedores trabalharam duro para estender os navegadores para que dados em outros formatos pudessem ser renderizados no computador cliente. Uma maneira de resolver o problema era permitir que o navegador, ao reconhecer um arquivo recebido de um tipo específico, iniciar um aplicativo separado na máquina cliente para renderizar o conteúdo. Desde que este aplicativo auxiliar tenha sido instalado no computador cliente, o navegador iniciará o programa e enviará o arquivo recebido para esse programa \citep{zammetti2007brief}.

A primeira solução para tornar a web mais dinâmica foi "Common Gateway Interface" (CGI), que permite a criação de programas que executem quando um usuário faz uma requisição. Porem, CGI não é a solução mais segura para criação de páginas web pois permite que seja executado um programa em seu sistema operacional e usuários maliciosos podem explorar isto com algum exploit\footnote{Uma sequência de comandos que tomam vantagem de um defeito, falha ou vulnerabilidade a fim de causar um comportamento acidental ou imprevisto} e executar operações indesejadas \citep{Asleson2006}.

Ainda segundo \citet{Asleson2006}, em maio de 1995 John Gage e Andreessen anunciam o nascimento da linguagem de programação Java. O navegador Netscape era dominante na época e ofereceria suporte para esta nova linguagem. Dentro de alguns meses, milhares de pessoas já haviam baixado o Java em seus computadores, abrindo novos caminhos para páginas web dinâmicas.

Applets\footnote{Applet é um pequeno software que executa uma atividade específica dentro de outro programa maior.} permitem que pequenas aplicações Java possam ser incluídas nas páginas web e executadas através da Java Virtual Machine (JVM). Apples são executadas no modelo de segurança de "caixa de areia" (sandbox), não podem carregar bibliotecas nativas e são tipicamente impedidas de ler ou gravar no disco \citep{Asleson2006}.

Ainda em maio de 1995, Brendan Eich, um funcionário da Netscape na época, desenvolveu uma linguagem de script em dez dias que se tornou conhecida como Mocha. Este nome original foi dado pelo fundador da Netscape. Pouco depois, o nome foi avaliado e renomeado como LiveScript. Mais tarde naquele ano em dezembro, a Netscape recebeu uma licença de marca registrada da Sun. Desta vez, o nome mudou para JavaScript \citep{neer2013history}.



------------------------------------------------------------------------------------
%pré ajax, iframes



% ajax polling

O termo Ajax é um acronimo para "Javascript assincrono e XML" \citep{Garrett2005} e descreve uma maneira de realizar uma comunicação HTTP a partir de uma aplicação Javascript em páginas web.


% long polling


% http chunk data stream


% SSE


%websockets

